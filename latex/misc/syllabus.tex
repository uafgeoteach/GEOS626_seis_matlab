% dvips -t letter syllabus.dvi -o syllabus.ps ; ps2pdf syllabus.ps
%
% UPDATE: dates (including report due date)
\documentclass[10pt,titlepage,fleqn]{article}

\usepackage{amsmath}
\usepackage{amssymb}
\usepackage{latexsym}
%\usepackage[round]{natbib}
\usepackage{xspace}
\usepackage{graphicx}
%\usepackage{epsfig}

%\usepackage{fancyhdr}
%\pagestyle{fancy}

\usepackage{color}

%=====================================================
%       SPACING COMMANDS (Latex Companion, p. 52)
%=====================================================

\usepackage{setspace}

\renewcommand{\baselinestretch}{1.1}

\textwidth 460pt
\textheight 700pt
\oddsidemargin 0pt
\evensidemargin 0pt

% see Latex Companion, p. 85
\voffset     -50pt
\topmargin     0pt
\headsep      20pt
\headheight    0pt
\footskip     30pt
\hoffset       0pt



% the ~ command keeps [Figure 2] always together,
% so that they are not separated by an end-of-line
\newcommand{\refEq}[1]{Equation~(\ref{#1})}
\newcommand{\refEqii}[2]{Equations~(\ref{#1}) and (\ref{#2})}
\newcommand{\refEqiii}[3]{Equations~(\ref{#1}), (\ref{#2}), and (\ref{#3})}
\newcommand{\refEqiiii}[4]{Equations~(\ref{#1}), (\ref{#2}), (\ref{#3}), and (\ref{#4})}
\newcommand{\refEqab}[2]{Equations (\ref{#1})--(\ref{#2})}
\newcommand{\refeq}[1]{Eq.~\ref{#1}}
\newcommand{\refeqii}[2]{Eqs.~\ref{#1} and \ref{#2}}
\newcommand{\refeqiii}[3]{Eqs.~\ref{#1}, \ref{#2}, and \ref{#3}}
\newcommand{\refeqiiii}[4]{Eqs.~\ref{#1}, \ref{#2}, \ref{#3}, and \ref{#4}}
\newcommand{\refeqab}[2]{Eqs.~\ref{#1}--\ref{#2}}

\newcommand{\refFig}[1]{Figure~\ref{#1}}
\newcommand{\refFigii}[2]{Figures~\ref{#1} and \ref{#2}}
\newcommand{\refFigiii}[3]{Figures~\ref{#1}, \ref{#2}, and \ref{#3}}
\newcommand{\refFigab}[2]{Figures~\ref{#1}--\ref{#2}}
\newcommand{\reffig}[1]{Fig.~\ref{#1}}
\newcommand{\reffigii}[2]{Figs.~\ref{#1} and \ref{#2}}
\newcommand{\reffigiii}[3]{Figs.~\ref{#1}, \ref{#2}, and \ref{#3}}
\newcommand{\reffigab}[2]{Figs.~\ref{#1}--\ref{#2}}

\newcommand{\refTab}[1]{Table~\ref{#1}}
\newcommand{\refTabii}[2]{Tables~\ref{#1} and \ref{#2}}
\newcommand{\refTabiii}[3]{Tables~\ref{#1}, \ref{#2}, and \ref{#3}}
\newcommand{\refTabab}[2]{Tables~\ref{#1}--\ref{#2}}

\newcommand{\refSec}[1]{Section~\ref{#1}}
\newcommand{\refSecii}[2]{Sections~\ref{#1} and \ref{#2}}
\newcommand{\refSeciii}[3]{Sections~\ref{#1}, \ref{#2}, and \ref{#3}}
\newcommand{\refSecab}[2]{Sections~\ref{#1}--\ref{#2}}

\newcommand{\refCha}[1]{Chapter~\ref{#1}}
\newcommand{\refApp}[1]{Appendix~\ref{#1}}

\newcommand{\refpg}[1]{p.~\pageref{#1}}

% spacing commands
\newcommand{\fgap}{\vspace{-14pt}}  % figure gap
\newcommand{\tgap}{\vspace{6pt}}    % table gap
\newcommand{\egap}{\vspace{10pt}}   % equation gap


% COMMANDS FROM JOHN'S ARTICLE
%\newcommand{\bmth}[1]{\mbox{\boldmath $ #1 $}}
%\newcommand{\m}[2]{m_{#1}^{(1)}m_{#2}^{(2)}}
%\newcommand{\sml}[1]{\mbox{\tiny $#1$}}
%\newcommand{\script}{\it}
%\newcommand{\dn}[1]{ {}{_{#1}} }
%\newcommand{\up}[1]{ {}{^{#1}} }

% fractions
\newcommand{\sfrac}[2]{\mbox{$\frac{{#1}}{{#2}\rule[-3pt]{0pt}{3pt}}$}}
\newcommand{\hlf}{\sfrac{1}{2}}
%\newcommand{\twothirds}{\sfrac{2}{3}}   % included in GJI
\newcommand{\fourthirds}{\sfrac{4}{3}}

\newcommand{\tp}{(\theta, \phi)}
\newcommand{\funa}[1]{{#1} \tp}
\newcommand{\funb}[1]{{#1}(\theta, \phi, t)}
\newcommand{\spo}[1]{{#1}_{\rm SP}}
\newcommand{\npo}[1]{{#1}_{\rm NP}}

\newcommand{\alm}{A_{lm}}
\newcommand{\blm}{B_{lm}}
\newcommand{\phom}{\psi_{\rm hom}}
\newcommand{\phet}{\psi_{\rm het}}
\newcommand{\chom}{c_{\rm hom}}
\newcommand{\chet}{c_{\rm het}}
\newcommand{\cmean}{c_{\rm mean}}
\newcommand{\cprem}{c_{\rm PREM}}
\newcommand{\dq}{\partial_{\theta}}
\newcommand{\dqq}{\partial_{\theta \theta}}
\newcommand{\latlon}[2]{$({#1}^\circ,\,{#2}^\circ)$}
\newcommand{\lalo}{(lat,~lon)}
\newcommand{\latf}{\bar{\theta}_f}

\newcommand{\lammin}{\Lambda_{\rm min}}
\newcommand{\lambar}{\bar{\Lambda}}

\newcommand{\dc}{\overline{\delta c}}
\newcommand{\pk}{\phi_k}
\newcommand{\pf}{\phi_f}
\newcommand{\tf}{\theta_f}
\newcommand{\cost}{\cos \theta}
\newcommand{\sint}{\sin \theta}
\newcommand{\plcost}{P_l(\cost)}
\newcommand{\plmt}{P_{lm}(\cost)}
\newcommand{\plmx}{P_{lm}(x)}
\newcommand{\coswlt}{\cos \omega_l t}
\newcommand{\sinwlt}{\sin \omega_l t}
\newcommand{\coswltau}{\cos\,\omega_l \tau}
\newcommand{\sinwltau}{\sin\,\omega_l \tau}
\newcommand{\delnu}{[\nabla^2 u]_{\rm nu}}
\newcommand{\delan}{[\nabla^2 u]_{\rm an}}
\newcommand{\dom}{{\tt dom}}
\newcommand{\abc}{(\alpha, \, \beta, \, \gamma)}
\newcommand{\const}{{\rm const}}
\newcommand{\andi}{\hspace{20pt} {\rm and} \hspace{20pt}}

\newcommand{\tper}[1]{$T$={#1}s}
\newcommand{\lmax}[1]{$lmax$={#1}}
\newcommand{\epsi}[1]{$\epsilon$={#1}}
\newcommand{\q}[1]{$q$={#1}}
\newcommand{\qmax}{q_{\rm max}}
\newcommand{\qmin}{q_{\rm min}}
\newcommand{\eg}{e.g.,\xspace}
\newcommand{\ie}{i.e.,\xspace}

\newcommand{\delmin}{\ensuremath{\Delta_{\rm{min}}}}
\newcommand{\delmax}{\ensuremath{\Delta_{\rm max}}}
\newcommand{\alphamax}{\ensuremath{\alpha_{\rm max}}}
\newcommand{\vd}[2]{\ensuremath{{\mathbf{#1}}_{#2}}}
\newcommand{\vu}[2]{\ensuremath{{\mathbf{#1}}^{#2}}}

%---------------------------------------------------------
% commands from Tromp
%---------------------------------------------------------

%bold lowercase roman letters

\newcommand{\ba}{\mbox{${\bf a}$}}
\newcommand{\bb}{\mbox{${\bf b}$}}
\newcommand{\bc}{\mbox{${\bf c}$}}
\newcommand{\bd}{\mbox{${\bf d}$}}
\newcommand{\be}{\mbox{${\bf e}$}}
\newcommand{\bef}{\mbox{${\bf f}$}}
\newcommand{\bg}{\mbox{${\bf g}$}}
\newcommand{\bh}{\mbox{${\bf h}$}}
\newcommand{\bi}{\mbox{${\bf i}$}}
\newcommand{\bj}{\mbox{${\bf j}$}}
\newcommand{\bk}{\mbox{${\bf k}$}}
\newcommand{\bl}{\mbox{${\bf l}$}}
\newcommand{\bem}{\mbox{${\bf m}$}}
\newcommand{\bn}{\mbox{${\bf n}$}}
\newcommand{\bo}{\mbox{${\bf o}$}}
\newcommand{\bp}{\mbox{${\bf p}$}}
\newcommand{\bq}{\mbox{${\bf q}$}}
\newcommand{\br}{\mbox{${\bf r}$}}
\newcommand{\bs}{\mbox{${\bf s}$}}
\newcommand{\bt}{\mbox{${\bf t}$}}
\newcommand{\bu}{\mbox{${\bf u}$}}
\newcommand{\bv}{\mbox{${\bf v}$}}
\newcommand{\bw}{\mbox{${\bf w}$}}
\newcommand{\bx}{\mbox{${\bf x}$}}
\newcommand{\by}{\mbox{${\bf y}$}}
\newcommand{\bz}{\mbox{${\bf z}$}}

%bold lowercase greek letters

\newcommand{\balpha}{\mbox{\boldmath $\bf \alpha$}}
\newcommand{\bbeta}{\mbox{\boldmath $\bf \beta$}}
\newcommand{\bgamma}{\mbox{\boldmath $\bf \gamma$}}
\newcommand{\bdelta}{\mbox{\boldmath $\bf \delta$}}
\newcommand{\bepsilon}{\mbox{\boldmath $\bf \epsilon$}}
\newcommand{\beps}{\mbox{\boldmath $\bf \varepsilon$}}
\newcommand{\bzeta}{\mbox{\boldmath $\bf \zeta$}}
\newcommand{\boeta}{\mbox{\boldmath $\bf \eta$}}
\newcommand{\btheta}{\mbox{\boldmath $\bf \theta$}}
\newcommand{\bkappa}{\mbox{\boldmath $\bf \kappa$}}
\newcommand{\blambda}{\mbox{\boldmath $\bf \lambda$}}
\newcommand{\bmu}{\mbox{\boldmath $\bf \mu$}}
\newcommand{\bnu}{\mbox{\boldmath $\bf \nu$}}
\newcommand{\bxi}{\mbox{\boldmath $\bf \xi$}}
\newcommand{\bpi}{\mbox{\boldmath $\bf \pi$}}
\newcommand{\bsigma}{\mbox{\boldmath $\bf \sigma$}}
\newcommand{\btau}{\mbox{\boldmath $\bf \tau$}}
\newcommand{\bupsilon}{\mbox{\boldmath $\bf \upsilon$}}
\newcommand{\bphi}{\mbox{\boldmath $\bf \phi$}}
\newcommand{\bchi}{\mbox{\boldmath $\bf \chi$}}
\newcommand{\bpsi}{\mbox{\boldmath$ \bf \psi$}}
\newcommand{\bomega}{\mbox{\boldmath $\bf \omega$}}
\newcommand{\bom}{\mbox{\boldmath $\bf \omega$}}
\newcommand{\brho}{\mbox{\boldmath $\bf \rho$}}

%bold uppercase roman letters

\newcommand{\bA}{\mbox{${\bf A}$}}
\newcommand{\bB}{\mbox{${\bf B}$}}
\newcommand{\bC}{\mbox{${\bf C}$}}
\newcommand{\bD}{\mbox{${\bf D}$}}
\newcommand{\bE}{\mbox{${\bf E}$}}
\newcommand{\bF}{\mbox{${\bf F}$}}
\newcommand{\bG}{\mbox{${\bf G}$}}
\newcommand{\bH}{\mbox{${\bf H}$}}
\newcommand{\bI}{\mbox{${\bf I}$}}
\newcommand{\bJ}{\mbox{${\bf J}$}}
\newcommand{\bK}{\mbox{${\bf K}$}}
\newcommand{\bL}{\mbox{${\bf L}$}}
\newcommand{\bM}{\mbox{${\bf M}$}}
\newcommand{\bN}{\mbox{${\bf N}$}}
\newcommand{\bO}{\mbox{${\bf O}$}}
\newcommand{\bP}{\mbox{${\bf P}$}}
\newcommand{\bQ}{\mbox{${\bf Q}$}}
\newcommand{\bR}{\mbox{${\bf R}$}}
\newcommand{\bS}{\mbox{${\bf S}$}}
\newcommand{\bT}{\mbox{${\bf T}$}}
\newcommand{\bU}{\mbox{${\bf U}$}}
\newcommand{\bV}{\mbox{${\bf V}$}}
\newcommand{\bW}{\mbox{${\bf W}$}}
\newcommand{\bX}{\mbox{${\bf X}$}}
\newcommand{\bY}{\mbox{${\bf Y}$}}
\newcommand{\bZ}{\mbox{${\bf Z}$}}

%bold uppercase greek letters

\newcommand{\bGamma}{\mbox{\boldmath $\bf \Gamma$}}
\newcommand{\bDelta}{\mbox{\boldmath $\bf \Delta$}}
\newcommand{\bLambda}{\mbox{\boldmath $\bf \Lambda$}}
\newcommand{\bXi}{\mbox{\boldmath $\bf \Xi$}}
\newcommand{\bPi}{\mbox{\boldmath $\bf \Pi$}}
\newcommand{\bSigma}{\mbox{\boldmath $\bf \Sigma$}}
\newcommand{\bUpsilon}{\mbox{\boldmath $\bf \Upsilon$}}
\newcommand{\bPhi}{\mbox{\boldmath $\bf \Phi$}}
\newcommand{\bPsi}{\mbox{\boldmath $\bf \Psi$}}
\newcommand{\bOmega}{\mbox{\boldmath $\bf \Omega$}}
\newcommand{\bOm}{\mbox{\boldmath $\bf \Omega$}}


%upper case sans serif letters

\newcommand{\ssA}{\mbox{${\sf A}$}}
\newcommand{\ssB}{\mbox{${\sf B}$}}
\newcommand{\ssC}{\mbox{${\sf C}$}}
\newcommand{\ssD}{\mbox{${\sf D}$}}
\newcommand{\ssE}{\mbox{${\sf E}$}}
\newcommand{\ssF}{\mbox{${\sf F}$}}
\newcommand{\ssG}{\mbox{${\sf G}$}}
\newcommand{\ssH}{\mbox{${\sf H}$}}
\newcommand{\ssI}{\mbox{${\sf I}$}}
\newcommand{\ssJ}{\mbox{${\sf J}$}}
\newcommand{\ssK}{\mbox{${\sf K}$}}
\newcommand{\ssL}{\mbox{${\sf L}$}}
\newcommand{\ssM}{\mbox{${\sf M}$}}
\newcommand{\ssN}{\mbox{${\sf N}$}}
\newcommand{\ssO}{\mbox{${\sf O}$}}
\newcommand{\ssP}{\mbox{${\sf P}$}}
\newcommand{\ssQ}{\mbox{${\sf Q}$}}
\newcommand{\ssR}{\mbox{${\sf R}$}}
\newcommand{\ssS}{\mbox{${\sf S}$}}
\newcommand{\ssT}{\mbox{${\sf T}$}}
\newcommand{\ssU}{\mbox{${\sf U}$}}
\newcommand{\ssV}{\mbox{${\sf V}$}}
\newcommand{\ssW}{\mbox{${\sf W}$}}
\newcommand{\ssX}{\mbox{${\sf X}$}}
\newcommand{\ssY}{\mbox{${\sf Y}$}}
\newcommand{\ssZ}{\mbox{${\sf Z}$}}

%lower case sans serif letters

\newcommand{\ssa}{\mbox{${\sf a}$}}
\newcommand{\ssb}{\mbox{${\sf b}$}}
\newcommand{\ssc}{\mbox{${\sf c}$}}
\newcommand{\ssd}{\mbox{${\sf d}$}}
\newcommand{\sse}{\mbox{${\sf e}$}}
\newcommand{\ssf}{\mbox{${\sf f}$}}
\newcommand{\ssg}{\mbox{${\sf g}$}}
\newcommand{\ssh}{\mbox{${\sf h}$}}
\newcommand{\ssi}{\mbox{${\sf i}$}}
\newcommand{\ssj}{\mbox{${\sf j}$}}
\newcommand{\ssk}{\mbox{${\sf k}$}}
\newcommand{\ssl}{\mbox{${\sf l}$}}
\newcommand{\ssm}{\mbox{${\sf m}$}}
\newcommand{\ssn}{\mbox{${\sf n}$}}
\newcommand{\sso}{\mbox{${\sf o}$}}
%\newcommand{\ssp}{\mbox{${\sf p}$}}    % conflict with some package
\newcommand{\ssq}{\mbox{${\sf q}$}}
\newcommand{\ssr}{\mbox{${\sf r}$}}
\newcommand{\sss}{\mbox{${\sf s}$}}
\newcommand{\sst}{\mbox{${\sf t}$}}
\newcommand{\ssu}{\mbox{${\sf u}$}}
\newcommand{\ssv}{\mbox{${\sf v}$}}
\newcommand{\ssw}{\mbox{${\sf w}$}}
\newcommand{\ssx}{\mbox{${\sf x}$}}
\newcommand{\ssy}{\mbox{${\sf y}$}}
\newcommand{\ssz}{\mbox{${\sf z}$}}

%bold unit vectors

%\newcommand{\bah}{\mbox{$\hat{\mbox{${\bf a}$}}$}}
%\newcommand{\bbh}{\mbox{$\hat{\mbox{${\bf b}$}}$}}
%\newcommand{\bdh}{\mbox{$\hat{\mbox{${\bf d}$}}$}}
%\newcommand{\beh}{\mbox{$\hat{\mbox{${\bf e}$}}$}}
%\newcommand{\bfh}{\mbox{$\hat{\mbox{${\bf f}$}}$}}
%\newcommand{\bgh}{\mbox{$\hat{\mbox{${\bf g}$}}$}}
%\newcommand{\bkh}{\mbox{$\hat{\mbox{${\bf k}$}}$}}
%\newcommand{\bnh}{\mbox{$\hat{\mbox{${\bf n}$}}$}}
%\newcommand{\bph}{\mbox{$\hat{\mbox{${\bf p}$}}$}}
%\newcommand{\brh}{\mbox{$\hat{\mbox{${\bf r}$}}$}}
%\newcommand{\bth}{\mbox{$\hat{\mbox{${\bf t}$}}$}}
%\newcommand{\bxh}{\mbox{$\hat{\mbox{${\bf x}$}}$}}
%\newcommand{\byh}{\mbox{$\hat{\mbox{${\bf y}$}}$}}
%\newcommand{\bzh}{\mbox{$\hat{\mbox{${\bf z}$}}$}}

\newcommand{\bah}{\mbox{\boldmath{$\hat{\rm a}$}}}
\newcommand{\bbh}{\mbox{\boldmath{$\hat{\rm b}$}}}
\newcommand{\bdh}{\mbox{\boldmath{$\hat{\rm d}$}}}
\newcommand{\beh}{\mbox{\boldmath{$\hat{\rm e}$}}}
\newcommand{\bfh}{\mbox{\boldmath{$\hat{\rm f}$}}}
\newcommand{\bgh}{\mbox{\boldmath{$\hat{\rm g}$}}}
\newcommand{\bih}{\mbox{\boldmath{$\hat{\rm i}$}}}
\newcommand{\bjh}{\mbox{\boldmath{$\hat{\rm j}$}}}
\newcommand{\bkh}{\mbox{\boldmath{$\hat{\rm k}$}}}
\newcommand{\bmh}{\mbox{\boldmath{$\hat{\rm m}$}}}
\newcommand{\bnh}{\mbox{\boldmath{$\hat{\rm n}$}}}
\newcommand{\bph}{\mbox{\boldmath{$\hat{\rm p}$}}}
\newcommand{\brh}{\mbox{\boldmath{$\hat{\rm r}$}}}
\newcommand{\bth}{\mbox{\boldmath{$\hat{\rm t}$}}}
\newcommand{\buh}{\mbox{\boldmath{$\hat{\rm u}$}}}
\newcommand{\bxh}{\mbox{\boldmath{$\hat{\rm x}$}}}
\newcommand{\byh}{\mbox{\boldmath{$\hat{\rm y}$}}}
\newcommand{\bzh}{\mbox{\boldmath{$\hat{\rm z}$}}}

\newcommand{\bBh}{\mbox{\boldmath{$\hat{\rm B}$}}}
\newcommand{\bCh}{\mbox{\boldmath{$\hat{\rm C}$}}}
\newcommand{\bFh}{\mbox{\boldmath{$\hat{\rm F}$}}}
\newcommand{\bGh}{\mbox{\boldmath{$\hat{\rm G}$}}}
\newcommand{\bHh}{\mbox{\boldmath{$\hat{\rm H}$}}}
\newcommand{\bLh}{\mbox{\boldmath{$\hat{\rm L}$}}}
\newcommand{\bRh}{\mbox{\boldmath{$\hat{\rm R}$}}}
\newcommand{\bNh}{\mbox{\boldmath{$\hat{\rm N}$}}}
\newcommand{\bSh}{\mbox{\boldmath{$\hat{\rm S}$}}}
\newcommand{\bTh}{\mbox{\boldmath{$\hat{\rm T}$}}}
\newcommand{\bPh}{\mbox{\boldmath{$\hat{\rm P}$}}}

%\newcommand{\bnuh}{\mbox{$\hat{\mbox{\boldmath $\bf \nu$}}$}}
%\newcommand{\bthetah}{\mbox{$\hat{\mbox{\boldmath $\bf \theta$}}$}}
%\newcommand{\betah}{\mbox{$\hat{\mbox{\boldmath $\bf \eta$}}$}}
%\newcommand{\bphih}{\mbox{$\hat{\mbox{\boldmath $\bf \phi$}}$}}
%\newcommand{\bDeltah}{\mbox{$\hat{\mbox{\boldmath $\bf \Delta$}}$}}
%\newcommand{\bThetah}{\mbox{$\hat{\mbox{\boldmath $\bf \Theta$}}$}}
%\newcommand{\bPhih}{\mbox{$\hat{\mbox{\boldmath $\bf \Phi$}}$}}
%\newcommand{\bsigmah}{\mbox{$\hat{\mbox{\boldmath $\bf \sigma$}}$}}

\newcommand{\balphah}{\mbox{\boldmath{$\hat{\alpha}$}}}

\newcommand{\bnuh}{\mbox{\boldmath{$\hat{\nu}$}}}
\newcommand{\bthetah}{\mbox{\boldmath{$\hat{\theta}$}}}
\newcommand{\betah}{\mbox{\boldmath{$\hat{\eta}$}}}
\newcommand{\bphih}{\mbox{\boldmath{$\hat{\phi}$}}}
\newcommand{\blambdah}{\mbox{\boldmath{$\hat{\lambda}$}}}

\newcommand{\bDeltah}{\mbox{\boldmath{$\hat{\Delta}$}}}
\newcommand{\bLambdah}{\mbox{\boldmath{$\hat{\Lambda}$}}}
\newcommand{\bThetah}{\mbox{\boldmath{$\hat{\Theta}$}}}
\newcommand{\bPhih}{\mbox{\boldmath{$\hat{\Phi}$}}}

\newcommand{\bsigmah}{\mbox{\boldmath{$\hat{\sigma}$}}}
\newcommand{\bgammah}{\mbox{\boldmath{$\hat{\gamma}$}}}
\newcommand{\bdeltah}{\mbox{\boldmath{$\hat{\delta}$}}}

\newcommand{\bthetahpr}{\mbox{$\hat{\mbox{\boldmath $\bf \theta$}}
            ^{\raise-.5ex\hbox{$\scriptstyle\prime$}}$}}
\newcommand{\bphihpr}{\mbox{$\hat{\mbox{\boldmath $\bf \phi$}}
            ^{\raise-.5ex\hbox{$\scriptstyle\prime$}}$}}
\newcommand{\bThetahpr}{\mbox{$\hat{\mbox{\boldmath $\bf \Theta$}}
            ^{\raise-.5ex\hbox{$\scriptstyle\prime$}}$}}
\newcommand{\bPhihpr}{\mbox{$\hat{\mbox{\boldmath $\bf \Phi$}}
            ^{\raise-.5ex\hbox{$\scriptstyle\prime$}}$}}

%other miscellaneous definitions (from Tromp)

\newcommand{\om}{\mbox{$\omega$}}
\newcommand{\Om}{\mbox{$\Omega$}}
\newcommand{\ep}{\mbox{$\varepsilon$}}
\newcommand{\p}{\mbox{$\partial$}}
\newcommand{\del}{\mbox{$\nabla$}}
\newcommand{\bdel}{\mbox{\boldmath $\nabla$}}
\newcommand{\bzero}{\mbox{${\bf 0}$}}
\renewcommand{\Re}{\mbox{${\rm R}$}}
\renewcommand{\Im}{\mbox{${\rm I}$}}
\newcommand{\real}{\mbox{$\Re{\rm e}$}}
\newcommand{\imag}{\mbox{$\Im{\rm m}$}}
\newcommand{\sqL}{\mbox{$k$}}
\newcommand{\sqLinv}{\mbox{$k^{-1}$}}
\newcommand{\nunl}{\mbox{${}_n\nu_l$}}
\newcommand{\tdot}{\,.\hspace{-0.98 mm}\raise.6ex\hbox{.}
                   \hspace{-0.98 mm}\raise1.2ex\hbox{.}\,}
\newcommand{\pvint}{\mathbin{\int{\mkern-19.2mu}-}}
\newcommand{\allspace}{\raise.25ex\hbox{$\scriptstyle\bigcirc$}}

%------------------------------------------------------------
% commands from Tromp,Tape,Liu (2005)

\newcommand{\frechet}{Fr\'{e}chet}
\newcommand{\pert}[1]{\delta \ln {#1}}
\newcommand{\pertB}[1]{\frac{\delta {#1}}{{#1}}}

\newcommand{\inter}[2]{{#1}$\sim${#2}$^\dagger$}
                                                                               
\newcommand{\shs}{SH$_{\rm S}$}
\newcommand{\shss}{SH$_{\rm SS}$}
\newcommand{\psvp}{P-SV$_{\rm P}$}
\newcommand{\psvs}{P-SV$_{\rm S}$}
\newcommand{\psvpp}{P-SV$_{\rm PP}$}
\newcommand{\psvss}{P-SV$_{\rm SS}$}
\newcommand{\psvpssp}{P-SV$_{\rm PS+SP}$}
                                                                               
\newcommand{\sdag}{\bs^\dagger}
\newcommand{\fdag}{\bef^\dagger}
\newcommand{\sbdag}{\bar{\bs}^\dagger}
\newcommand{\fbdag}{\bar{\bef}^\dagger}
\newcommand{\Ddag}{\bD^\dagger}
\newcommand{\Dbdag}{\bar{\bD}^\dagger}
                                                                               
%\newcommand{\ka}{K_\alpha}           % \kabr
%\newcommand{\kb}{K_\beta}            % \kbar
%\newcommand{\krp}{K_{\rho'}}         % \krab
%\newcommand{\kk}{K_\kappa}           % \kkmr
%\newcommand{\km}{K_\mu}              % \kmkr
%\newcommand{\kr}{K_\rho}             % \krkm

\newcommand{\kabr}{K_{\alpha(\beta\rho)}}
\newcommand{\kbar}{K_{\beta(\alpha\rho)}}
\newcommand{\krab}{K_{\rho(\alpha\beta)}}
\newcommand{\kkmr}{K_{\kappa(\mu\rho)}}
\newcommand{\kmkr}{K_{\mu(\kappa\rho)}}
\newcommand{\krkm}{K_{\rho(\kappa\mu)}}
\newcommand{\kcbr}{K_{c(\beta\rho)}}
\newcommand{\kbcr}{K_{\beta(c\rho)}}
\newcommand{\krcb}{K_{\rho(c\beta)}}
                                                                               
%\newcommand{\kba}{\bar{K}_\alpha}    % \kbabr
%\newcommand{\kbb}{\bar{K}_\beta}     % \kbbar
%\newcommand{\kbrp}{\bar{K}_{\rho'}}  % \kbrab
%\newcommand{\kbk}{\bar{K}_\kappa}    % \kbkmr
%\newcommand{\kbm}{\bar{K}_\mu}       % \kbmkr
%\newcommand{\kbr}{\bar{K}_\rho}      % \kbrkm

\newcommand{\kbabr}{\bar{K}_{\alpha(\beta\rho)}}
\newcommand{\kbbar}{\bar{K}_{\beta(\alpha\rho)}}
\newcommand{\kbrab}{\bar{K}_{\rho(\alpha\beta)}}
\newcommand{\kbkmr}{\bar{K}_{\kappa(\mu\rho)}}
\newcommand{\kbmkr}{\bar{K}_{\mu(\kappa\rho)}}
\newcommand{\kbrkm}{\bar{K}_{\rho(\kappa\mu)}}
\newcommand{\kbcbr}{\bar{K}_{c(\beta\rho)}}
\newcommand{\kbbcr}{\bar{K}_{\beta(c\rho)}}
\newcommand{\bkrcb}{\bar{K}_{\rho(c\beta)}}

%------------------------------------------------------------
% commands added post-2004

\newcommand{\abr}{$\alpha$-$\beta$-$\rho$}
\newcommand{\cbr}{$c$-$\beta$-$\rho$}
\newcommand{\kmr}{$\kappa$-$\mu$-$\rho$}

% SCEC FILE
\newcommand{\mone}{u_{0x}}
\newcommand{\mtwo}{u_{0y}}

% CONTINUUM TABLES
\newcommand{\delx}{\nabla_x}
\newcommand{\delX}{\nabla_X}
\newcommand{\delxv}{\nabla_x {\bf v}}
\newcommand{\delu}{\nabla {\bf u}}
\newcommand{\detr}[1]{ {\rm det}({#1}) }

% STATS TABLES
\newcommand{\summy}{ \sum_{i=1}^{n} }
\newcommand{\bhi}{\hat{\beta}_0}
\newcommand{\bhs}{\hat{\beta}_1}

\newcommand{\cov}{{\rm cov}}
\newcommand{\var}{{\rm var}}
\newcommand{\erf}{{\rm erf}}
\newcommand{\Var}{{\rm Var}}

\newcommand{\tband}[2]{\mbox{$T = [{#1}\,{\rm s}-{#2}\,{\rm s}]$}}

% earthquake magnitudes
\newcommand{\magB}[1]{$m_{\rm B}\,${#1}}
\newcommand{\magb}[1]{$m_{\rm b}\,${#1}}
\newcommand{\magw}[1]{$M_{\rm w}\,${#1}}
\newcommand{\mags}[1]{$M_{\rm s}\,${#1}}
\newcommand{\magt}[1]{$M_{\rm t}\,${#1}}
\newcommand{\magM}[1]{$M\,${#1}}

\newcommand{\mw}{M_{\rm w}}
\newcommand{\hdur}{\tau_{\rm h}}

% model vector
\newcommand{\smodel}[1]{{\bf m}_{\rm {#1}}}

%------------------------------------------------------------
% from qinya's new commands

% cal font
\newcommand{\cF}{\mbox{$\cal F$}}
\newcommand{\cC}{\mbox{$\cal C$}}

% variations
\newcommand{\dvphi}{\delta\varphi}
\newcommand{\dbS}{\delta\bS}
\newcommand{\dbs}{\delta\bs}
\newcommand{\dbm}{\delta\bem}
\newcommand{\dbc}{\delta\bc}
\newcommand{\dbf}{\delta\bef}
\newcommand{\dm}{\delta m}
\newcommand{\df}{\delta f}
\newcommand{\dF}{\delta F}
\newcommand{\drho}{\delta\rho}

% it font
\newcommand{\iL}{\mathit{L}}
\newcommand{\iF}{\mathit{F}}

\newcommand{\bFz}{\mathbf{F_0}}
\newcommand{\sd}{\dot{s}}
\newcommand{\sdd}{\ddot{s}}

\newcommand{\vp}{V_{\rm P}}
\newcommand{\vs}{V_{\rm S}}
\newcommand{\vb}{V_{\rm B}}

% attenuation
\newcommand{\qp}{Q_{\rm P}}
\newcommand{\qs}{Q_{\rm S}}

% data and synthetics
% amplitude anomalies
\newcommand{\Aamp}{A}
\newcommand{\Aobs}{\Aamp_{\rm obs}}
\newcommand{\Asyn}{\Aamp_{\rm syn}}
% data and synthetics as scalar functions (of time)
\newcommand{\uobs}{u}
\newcommand{\usyn}{s}
% data and synthetics as a discrete vector of points
\newcommand{\uobsb}{\bu}
\newcommand{\usynb}{\bs}

%------------------------------------------------------------
% 2009 PhD thesis

\newcommand{\eq}{\begin{equation}}
\newcommand{\en}{\end{equation}}
\newcommand{\eqa}{\begin{eqnarray}}
\newcommand{\ena}{\end{eqnarray}}

% convention for variables
\newcommand{\misfitvar}{F}
\newcommand{\misfitt}{\misfitvar^{\rm T}}
\newcommand{\misfitw}{\misfitvar^{\rm W}}
%\newcommand{\nparm}{M}           % number of data
%\newcommand{\ndata}{N}           % number of data
\newcommand{\nrec}{R}            % number of stations
\newcommand{\irec}{r}            % staation index
\newcommand{\nsrc}{S}            % number of sources
\newcommand{\isrc}{s}            % source index
\newcommand{\ipick}{p}            % pick index
\newcommand{\ncmp}{C}            % number of components
\newcommand{\normt}{M_{\rm T}}   % normalization for DT adjoint source
\newcommand{\norma}{M_{\rm A}}   % normalization for DA adjoint source
\newcommand{\normtr}[1]{M_{{\rm T}{#1}}}   % normalization for DT adjoint source -- with subscript
\newcommand{\normar}[1]{M_{{\rm A}{#1}}}   % normalization for DA adjoint source -- with subscript

\newcommand{\rmd}{\mathrm{d}}        % roman d for dV in integrals 
\newcommand{\nglob}{{N_{\rm glob}}} 
\newcommand{\mray}{\bem^{\rm ray}}
\newcommand{\mker}{\bem^{\rm ker}}

%------------------------------------------------------------
% subspace paper

\newcommand{\rms}{\mathrm{s}}
\newcommand{\db}{\overline{d}}
\newcommand{\tssT}{\mathsf{T}}
\newcommand{\tssm}{\mathsf{m}}
\newcommand{\tssd}{\mathsf{d}}
\newcommand{\sslambda}{\mathsf{\lambda}}
\newcommand{\ssgamma}{\mathsf{\gamma}}
\newcommand{\ssmu}{\mathsf{\mu}}
\newcommand{\ssbeta}{\mathsf{\beta}}
\newcommand{\ssalpha}{\mathsf{\alpha}}

\newcommand{\ascent}{\gamma}
\newcommand{\bascent}{\bgamma}
\newcommand{\mprior}{\bem_{\rm prior}}
\newcommand{\mpost}{\bem_{\rm post}}
\newcommand{\mtarget}{\bem_{\rm target}}
%\newcommand{\Cprior}{\bC_{\tssm_0}}
\newcommand{\Cds}{\bC_{{\rm D}_s}}

\newcommand{\covd}{\bC_{\rm D}}
\newcommand{\covm}{\bC_{\rm M}}
\newcommand{\covdi}{\bC_{\rm D}^{-1}}
\newcommand{\covmi}{\bC_{\rm M}^{-1}}
\newcommand{\cprior}{\bC_{\rm prior}}
%\newcommand{\Cprior}{\bC_{\rm M}}
\newcommand{\cpriori}{\bC_{\rm prior}^{-1}}
\newcommand{\cpost}{\bC_{\rm post}}
\newcommand{\cposti}{\bC_{\rm post}^{-1}}

\newcommand{\grad}{\hat{\gamma}}
\newcommand{\bgrad}{\bgammah}
\newcommand{\mvec}{\bem}
\newcommand{\hessM}{\bHh}
\newcommand{\hessD}{\bD}
\newcommand{\muvec}{\Delta\bmu}
\newcommand{\Cpost}{\bC_{\rm post}}
%\newcommand{\Cpost}{\bC_{\tssm}}
\newcommand{\Cd}{\bC_{\rm D}}

%\newcommand{\misfitvartilde}{\widetilde{\misfitvar}}
\newcommand{\misfitvartilde}{\misfitvar}
\newcommand{\gradtilde}{\widetilde{\hat{\gamma}}}
%\newcommand{\bgradtilde}{\widetilde{\bgammah}}
\newcommand{\bgradtilde}{\bgammah}
\newcommand{\mvectilde}{\bem}
%\newcommand{\hessMtilde}{\widetilde{\bHh}}
\newcommand{\hessMtilde}{\bHh}
\newcommand{\hessDtilde}{\widetilde{\bD}}
\newcommand{\muvectilde}{\Delta\widetilde{\bmu}}
%\newcommand{\Cposttilde}{\widetilde{\bC}_{\rm post}}
%\newcommand{\Cposttilde}{\widetilde{\bC}_{\tssm}}
\newcommand{\Cposttilde}{\bC_{\rm post}}
\newcommand{\Cdtilde}{\widetilde{\bC}_{\rm D}}
\newcommand{\Btilde}{\widetilde{\bB}}

%------------------------------------------------------------

\newcommand{\frange}[2]{\mbox{{#1}--{#2}~Hz}}
\newcommand{\trange}[2]{\mbox{{#1}--{#2}~s}}
\newcommand{\lrange}[2]{\mbox{$l$={#1}--{#2}}}

% Alessia's paper
%\newcommand{\stga}{Stage~A}
%\newcommand{\stgb}{Stage~B}
%\newcommand{\stgc}{Stage~C}
%\newcommand{\stgd}{Stage~D}
%\newcommand{\stge}{Stage~E}

% names with accents
\newcommand{\vala}{Hj\"orleifsd\'ottir}
\newcommand{\moho}{Mohorovi\v{c}i\'{c}}

% matrix trace
\newcommand{\tra}[1]{\mathrm{tr}\left(#1\right)}

\newcommand{\nn}{\nonumber}

\newcommand{\rref}{\mathrm{rref}}
\newcommand{\proj}{\mathrm{proj}}
\newcommand{\rank}{\mathrm{rank}}
\newcommand{\cond}{\mathrm{cond}}

\newcommand{\fnyq}{f_{\rm Nyq}}

%------------------------------------------------------------
% moment tensor papers by Vipul, Celso, etc

\newcommand{\gd}[2]{$\gamma~{#1}^\circ$, $\delta~{#2}^\circ$}
% XXX probably we want to change sdr to be srd (BUT BE CAREFUL):
\newcommand{\sdr}[3]{strike~${#1}^\circ$, dip~${#2}^\circ$, rake~${#3}^\circ$} 
\newcommand{\mmin}{\ensuremath{M_0}}     % moment tensor global minimum
\newcommand{\mmax}{\ensuremath{M_x}}     % moment tensor global maximum
\newcommand{\vr}{\ensuremath{\mathit{VR}}}
\newcommand{\vrmax}{\ensuremath{\vr_{\max}}}

%------------------------------------------------------------


\graphicspath{
  {figures/}
}

\newcommand{\xxa}{1/17}
\newcommand{\xxb}{1/22}
\newcommand{\xxc}{1/24}
\newcommand{\xxd}{1/29}
\newcommand{\xxe}{1/31}
\newcommand{\xxf}{2/5}
\newcommand{\xxg}{2/7}
\newcommand{\xxh}{2/12}
\newcommand{\xxi}{2/14}
\newcommand{\xxj}{2/19}
\newcommand{\xxk}{2/21}
\newcommand{\xxl}{2/26}
\newcommand{\xxm}{2/28}
\newcommand{\xxn}{3/5}
\newcommand{\xxo}{3/7}
\newcommand{\xxp}{3/19}
\newcommand{\xxq}{3/21}
\newcommand{\xxr}{3/26}
\newcommand{\xxs}{3/28}
\newcommand{\xxt}{4/2}
\newcommand{\xxu}{4/4}
\newcommand{\xxv}{4/9}
\newcommand{\xxw}{4/11}
\newcommand{\xxx}{4/16}
\newcommand{\xxy}{4/18}
\newcommand{\xxz}{4/23}
\newcommand{\xxaa}{4/25}
\newcommand{\xxbb}{4/30}

%=====================================================
\begin{document}
%=====================================================

\begin{tabular}{cc}
\includegraphics[width=8cm]{/home/admin/share/datalib/logos/UAF/UAF/UAFLogo_A_black_horiz.eps} &
\end{tabular}

\bigskip\noindent
{\bf \em QUICK REFERENCE: Section 8 contains the calendar of topics and deadlines.}

\medskip\noindent Last compiled: \today

\begin{enumerate}
%------------
\item {\bf Course information.}

\begin{tabular}{ll}
GEOS 626  & {\bf Applied Seismology}, 4 credits (3+3), Spring 2018 \\
Lecture:  & Wednesday, 8:30--10:30, 310B Elvey\\
Lab:      & Monday, 9:00--13:00, 307 Elvey \\
Prerequisites: & MATH 314 (Linear Algebra) and MATH 252 (Calculus III) \\
%               & GEOS 604 (Seismology) is a recommended prerequisite
\end{tabular}

%\noindent
%The 3+3 designation refers to the number of hours in lecture per week (3) and the number of hours in lab each week (3). As a UA guideline, for each 1 hour of lecture a student is expected to spend 2~hours outside of class on work related to the course. So for this class, you are expected to spend a total of 12 hours per week on the class.

%------------
\item {\bf Instructor information.}

\begin{tabular}{ll}
Instructor: & {\bf Carl Tape} \\
Office: & 413D Elvey (Geophysical Institute) \\
Email: & \verb+ctape@alaska.edu+ \\
Phone: & (907) 474-5456 \\
Office hours: & Friday, 10:00--11:00, or by appointment \\
\end{tabular}

%------------
\item {\bf Course materials.}


\begin{enumerate}

\item {\bf Textbooks.} The recommended (R) and supplemental (S) textbooks are listed in the following table; bibliographic details are listed at the end of this syllabus.
%``Software'' lists the software (if any) used in examples within each book.
The  Geophysical Institute's Mather library is located in the IARC/Akasofu building.

\begin{tabular}{l|c|c|c|c|c|c}
\hline
     &          & & \multicolumn{4}{c}{Availability}    \\ \cline{4-7}
Textbook & R & S  & UAF       & Mather  &     & UAF     \\
     &          &           & bookstore & reserve & PDF & e-book  \\ \hline
\cite{SteinWysession} Stein and Wysession   & X &      & & X & & X \\ \hline
\cite{ShearerE2} Shearer                    & X &      & & X & & \\ \hline
\cite{Igel} Igel                            & X &      & & X & & \\ \hline
\cite{KennettV1,KennettV2,DT,LayWallace1995,AkiRichardsE2,Malvern,AsterE2,Fichtner} & & X & & X & & \\ \hline
\end{tabular} \\

\item For an excellent overview of the field of seismology, consider reading \cite{WHKLeeA,WHKLeeB}, both on (permanent) reserve at Mather library.

\item Journal articles assigned as reading will be available as PDFs via the UAF google drive. All students must access the drive with their \verb+alaska.edu+ email address.

\item Students will need computers for their homework. General-use computers in UAF labs will be made available to students if needed.

\item Matlab will be the primary computational program for the course.
Matlab is available via a UAF-wide license for graduate students and faculty.
Students are welcome to use alternative programs such as python for homework problems; however several labs and examples are in Matlab.
%Matlab is available on computers in the student computer lab in Reichardt 316. If you need access to the lab, please follow the online lab instructions at \verb+www.uaf.edu/geology/facilities/computer/computer-labs/+, and contact Chris Wyatt to set up an account (\verb+wcwyatt@alaska.edu+). Note that it may be possible to obtain Matlab via OIT (\verb+www.alaska.edu/oit/+).

\end{enumerate}

%------------
\item {\bf Course description.}

Seismology combines observational data (seismograms) with numerical modeling methods to obtain powerful inferences about earthquake sources and the three-dimensional structure of Earth's interior. {\em Applied Seismology} will provide essential training for students' interested in academic, industrial, or governmental careers in seismology.

{\em Catalog description}: Presentation of modeling techniques for earthquakes and Earth structure using wave propagation algorithms and real seismic data. Covers several essential theories and algorithms for applications in seismology, as well as the basic tools needed for processing and using recorded seismograms. Topics include the seismic wavefield (body waves and surface waves), earthquake moment tensors, earthquake location, and seismic tomography. Assignments require familiarity with linear algebra and computational tools such as Matlab.

%------------
\item {\bf Course goals.}

We will explore the study of earthquakes and Earth's interior structure using seismological theories and algorithms. The underlying physical phenomenon we will examine is the seismic wavefield: the time-dependent, space-dependent elastic waves that originate at an earthquake source (for example, a fault slips) and propagate though the heterogeneous Earth structure, then are finally recorded as time series at seismometers on Earth's surface. Students will examine real seismic data and use computational models to estimate properties about earthquake source and Earth structure. Students will acquire practical, advanced seismological training that will prepare them for seismological investigations in the future, whether in academic, industry, or government jobs.

%------------
\item {\bf Student learning outcomes.}

Upon completion of this course, students should be able to:
%
\begin{enumerate}
\item Understand the relevant temporal, spatial, and magnitude scales in the field of seismology.
\item Describe the physical quantities that govern seismic wave propagation.
\item Describe the seismic phases that arise in a regional or global layered Earth model.
\item Describe the seismic moment tensor, the fundamental model of an earthquake source.
\item Understand the basic framework of inverse problems within the context of seismology.
\item Describe several different seismological tools that can be used to investigate an individual earthquake.
\item Understand the connection between earthquakes, continental deformation, and plate tectonics.
\item Understand the distinction between one-dimensional and three-dimensional Earth structure, and how this affects theory and algorithms in seismology.
\item Read seismological journal articles and summarize the content efficiently.
\item Write, improve, and run simple computational algorithms in Matlab.
\item Plot and manipulate recorded seismograms.
\end{enumerate}

%------------
\item {\bf Instructional methods.}

\begin{enumerate}
%\item Assignments and grades (along with general course information and handouts) will be posted on Blackboard: \verb+classes.uaf.edu+.
\item General course information, assignments, and handouts will be posted on the class website or in the class google drive.
\item Lectures: 3 hours per week.
%\item Each student is expected to lead one brief discussion and review of an assigned journal article.
\item Labs: Computational laboratory sessions (3 hours per week) include dedicated exercises that provide technical training for homework problems.
\end{enumerate}

%------------

\item {\bf Course calendar (tentative).}

\noindent {\bf Some Important Dates:}

\begin{tabular}{lll}
\hline
First day of UAF classes:                           & Tuesday & January 16 \\
First GEOS 626 meeting:                             & Wednesday & January 17 \\
Last day to add class:                              & Friday & January 26 \\
Last day to drop class:                             & Friday & January 26 \\
Last day for student- or faculty-initiated withdraw: & Friday & March 30 \\
Last GEOS 626 meeting:                              & Monday & April 30 \\
Last day of UAF classes:                            & Monday & April 30 \\
Finals week:                                        & Tues-Sat & May 1--5 \\
%Research project report and presentation:              & & TBD (May 6--9) \\
\hline
\end{tabular}

\pagebreak

\hspace{-1cm}
\begin{tabular}{cl|l|l|ll}
\hline
   & Date    & Topic & Reading & \multicolumn{2}{c}{Homework} \\
   &         &       & Due$^\dagger$     & Due & Assigned \\ \hline\hline
1  & \xxa\   & Seismology: past, present, future & SW1, S1 & --- & HW-1 \\
\hline
2  & \xxb\   & introduction to seismology & SW1, S1 & & \\
3  & \xxc\   & linear algebra and vectors & SW-A, S-B & & \\
   &         & LAB (\verb+doc_startup.pdf+): Linux and Matlab & & & \\
\hline
4  & \xxd\   & linear algebra and vectors & \verb+notes_matrix.pdf+ & HW-1 & HW-2 \\
5  & \xxe\   & seismic moment tensor & & \\
   &         & LAB (\verb+beachball+): seismic moment tensors & & & \\
\hline
6  & \xxf\   & seismic moment tensor & SW4.4, S9 & HW-2 & HW-3 \\
7  & \xxg\   & seismic moment tensor & \verb+notes_mt_626.pdf+ & & \\
   &         & LAB (\verb+mt+): seismic moment tensors & & & \\
\hline
8  & \xxh\   & Fourier transform & notes, SW6, S-E & HW-3 & HW-4\\
9  & \xxi\   & LAB (\verb+fft+): Fourier transform, seismic spectra & & \\
   &         & LAB (\verb+response+): instrument response & & \\
\hline
10 & \xxj\   & the 2004 Sumatra earthquake & \cite{Lay2005,Ammon2005,Park2005,Ni2005} & & \\
11 & \xxk\   & LAB: Sumatra earthquake (HW-4) & & \\
% AT WHAT STAGE DO WE NEED GEOTOOLS FOR doc_startupB.pdf?
\hline
12 & \xxl\   & processing seismic data & SW6, \cite{ReyesWest2011} & HW-4 & HW-5 \\
13 & \xxm\   & waves on a string & SW2 & & \\
   &         & LAB: modes of a spherical shell  & & \\
\hline
14 & \xxn\   & normal modes: theory and observations & SW2.9, S8.6 & HW-5 & HW-6 \\
15 & \xxo\   & normal modes: theory and observations & DT10.5 & & \\
   &         & LAB (\verb+seismo_rs+): analyzing seismic data & & & \\
\hline\hline
   &         & \multicolumn{4}{c}{SPRING BREAK} \\
   &         & \multicolumn{4}{c}{SPRING BREAK} \\
\hline\hline
16 & \xxp\   & review HW4 and HW5 & & & \\
17 & \xxq\   & seismic data analysis & & & \\
   &         & LAB (\verb+sumatra+): modes of Sumatra  & & & \\
\hline
18 & \xxr\   & the 2004 Sumatra earthquake & \cite{Lay2005,Ammon2005,Park2005,Ni2005} & HW-6 & HW-7 \\
19 & \xxs\   & great earthquakes since 2000 & & & \\
   &         & LAB: Love waves for layer-over-halfspace & & & \\
\hline
20 & \xxt\   & surface waves: theory and observations & SW2.7-2.8, S8 & HW-7 & HW-8 \\
21 & \xxu\   & surface waves: theory and observations & SW2.7-2.8, S8 & & \\
   &         & LAB: & & & \\
\hline
22 & \xxv\   & introduction to inverse problems & SW7 & HW-8 & HW-9 \\
23 & \xxw\   & introduction to least squares    & SW7, notes & & \\
   &         & LAB (\verb+linefit+): least squares & & & \\
\hline
24 & \xxx\   & seismic tomography: global & S5, SW7.3 & HW-9 & HW-10 \\
25 & \xxy\   & seismic tomography: crustal & SW3.2-3.3 & & \\
   &         & LAB: seismic tomography & & & \\
\hline
26 & \xxz\   & least-squares inverse theory & T3 & & \\
27 & \xxaa\  & iterative methods & T6.22 & & \\
   &         & LAB (\verb+newton+): Newton method & & & \\
\hline
28 & \xxbb\  & adjoint methods in seismology & \cite{LiuTromp2006,Tape2007} & HW-10 & \\
   &         & finite source models & S9.8, WS4.5 & & \\
   &         & LAB (\verb+dispersion+): surface wave dispersion & & & \\
\hline
%28 &      & seismology in the oil industry & S7, WS3.3 & & \\
%29 &      & seismic monitoring for nuclear activity & \cite{BowersSelby2005} & & \\
%   &      & LAB: TBD & & & \\
%\hline
% & & May-XX & \multicolumn{4}{c}{FINAL PROJECT PRESENTATION} \\
\hline
\end{tabular} \\
$^\dagger$SW = Ref.~\cite{SteinWysession}; S = Ref.~\cite{ShearerE2}; DT = Ref.~\cite{DT}; T = Ref.~\cite{Tarantola2005} \\
For example, ``SW2.9'' means Section 2.9 of Stein and Wysession (Ref.~\cite{SteinWysession}); ``S-E'' means Appendix~E of Shearer (Ref.~\cite{ShearerE2})

% LIST OF TOPICS
% Basic analysis and processing of seismograms & & PS-1 & PS-2 \\ \hline
% Continuum mechanics & DT2.6 & & \\
% Equations of motion & DT3, SW2, S2 & PS-2 & PS-3 \\ \hline
% Solving the wave equation (3D) & DT2 & & \\
% Solving the wave equation (1D and 2D) & SW2, S3 & PS-3 & PS-4 \\ \hline
% Normal modes: theory and observations & SW2.9, S8.6, DT10.5 & & \\
% Surface waves: theory and observations & SW2.7-2.8, S8 & PS-4 & PS-5 \\ \hline
% Body waves, reflection, and transmission & S4, SW3 & & \\
% Waveform modeling & SW4.3 & PS-5 & PS-6 \\ \hline
% Wavefield modeling & \cite{NissenMeyer2007axi,KomaTromp2002a,KomaTromp2002b} & & \\
% Finite-frequency sensitivity kernels & \cite{Hung2000a,Hung2000b} & PS-6 & PS-7 \\ \hline
% Ambient-noise tomography & \cite{Shapiro2005,Tromp2010} & & final project \\
% Preliminary Reference Earth Model & \cite{PREM}, DT8.2 & PS-7 & PS-8 \\ \hline
% Forward problems and inverse problems & \cite{PREM} & & \\
% Earthquake location & SW4, S9& PS-8 & PS-9 \\ \hline
% Seismic moment tensor & SW4.4, S9& & \\
% Finite source models & S9.8, WS4.5 & PS-9 & PS-10 \\ \hline
% Seismic tomography: global & S5, SW7.3 & & \\
% Seismic tomography: crustal & SW3.2-3.3 & PS-10 & PS-11 \\ \hline
% Anisotropy and attenuation & SW3.6-3.7, S6.6,11.3 & & \\
% Adjoint methods in seismology & \cite{LiuTromp2006,Tape2007} & PS-11 & final project \\ \hline
% Finite source inversion & S9.8, SW4.5 & & final project \\
% Seismology, geodesy, and deformation & WS5 & & final project \\ \hline
% Seismology of volcanoes & \cite{McNutt2005} & & final project \\
% Seismology of glaciers & \cite{TsaiEkstrom2007,MWest2010} & & final project \\ \hline
% Seismology in the oil industry & S7, WS3.3 & & final project \\
% Seismic monitoring for nuclear activity & \cite{BowersSelby2005} & REPORT & \\ \hline

%------------
\pagebreak
\item {\bf Course policies.}

../../../inverse/latex/misc/syllabus_policies.tex

%------------
%\pagebreak
\item {\bf Evaluation.}

\begin{enumerate}
%\item For students in the M.S. or Ph.D. program, you must receive a C (note: not C-) or higher for this course for it to count toward your degree requirements.

\item Grading is based on:

\begin{tabular}{|l|l|}
\hline
10\% & Lab participation and lab assignments \\ \hline
90\% & Homework assignments \\ \hline
%20\% & Individual research project \\ \hline
\end{tabular}

\bigskip
\item Overall course grades are based on the following:

\begin{tabular}{|l|c|l|}
\hline
\hline
A & $x \ge 93$ & excellent performance:  \\
A-- & $90 \le x < 93$ & student demonstrates deep understanding of the subject \\ \hline
B+ & $87 \le x < 90$ & strong performance: \\ 
B  & $83 \le x < 87$ & student demonstrates strong understanding of the subject, \\ 
B-- & $80 \le x < 83$ & but the work lacks the depth and quality needed for an `A' \\ \hline
C+ & $77 \le x < 80$ & mediocre performance: \\
C  & $73 \le x < 77$ & student demonstrates comprehension of some\\
C-- & $70 \le x < 73$ & essential concepts only \\ \hline
D  & $60 \le x < 70$ & poor performance: \\ 
   &                 & student demonstrates poor comprehension of concepts \\ \hline
F  & $x < 60$ & Failure to complete work with 60\% quality \\ \hline
\end{tabular}

\bigskip
\item {\bf Research Project.}

Students have the option of of substituting a research project for any 2 homework assignments.
%
\begin{itemize}
%is worth $30\%$ of your final grade.
\item It is due in the form of a report on Monday, April 30.
%and presentation on Thursday, May 3.
%\item You have the choice to do the final three problem sets (Sumatra data, moment tensors, seismic tomography) in place of the final project. In this case the three problem sets will account for $30\%$ of your final grade.

\item The theme of the project is {\bf Exploration of the Seismic Wavefield}.
\item The project should contain three components:
\begin{enumerate}
\item a review of essential literature on the topic
\item a detailed explanation of what facet of the seismic wavefield is represented by the project
\item a moderate level of applied analysis, either through modeling or examination of data
\end{enumerate}
The instructor will base his evaluation of the project on these three components, weighted equally.
\item The report should contain no more than 6 pages of single-spaced text (not including references) and 4 pages of figures. The report will be written in manuscript-submission style and format, using the guidelines for {\em Geophysical Research Letters}.
\end{itemize}
%
Students are welcome to propose topics to the instructor. Here are some possibilities:
%
\begin{itemize}
\item Exploration of air-solid-topography coupling of wave propagation. Code: SPECFEM2D
\item Generation of 1D synthetic seismograms using normal modes or axisymmetric spectral-element method. Code: MINEOS, AXISEM.
\item Seismic moment tensor inversion of regional earthquakes. Code: FK, CAP.
\item Eigenfunctions and eigenfrequencies for radial and spheroidal modes. Code: Matlab.
\item Implementation and application of some semi-standard seismological software package:
\verb+http://www.orfeus-eu.org/Software/softwarelib.html+
\item Investigation of variability of finite source models:
\verb+http://www.seismo.ethz.ch/static/srcmod/+
\item Resolvability of the isotropic component of source mechanisms using 2D synthetic experiments. Code: SPECFEM2D.
\end{itemize}

\end{enumerate}

%------------
\item {\bf Support Services.}

The instructor is available by appointment for additional assistance outside session hours. UAF has many student support programs, including the Math Hotline (1-866-UAF-MATH; 1-866-6284) and the Math and Stat Lab in Chapman building (see \verb+www.uaf.edu/dms/mathlab/+ for hours and details).

%------------
\item {\bf Disabilities Services.}

The Office of Disability Services implements the Americans with Disabilities Act (ADA), and it ensures that UAF students have equal access to the campus and course materials. The Geophysics Program will work with the Office of Disability Services (208 Whitaker, 474-5655) to provide reasonable accommodation to students with disabilities.

%=====================================================

\pagebreak
\item {\bf References listed in syllabus.}

\renewcommand{\refname}{}

\vspace{-1.4cm}

\begin{spacing}{1.0}
%\bibliographystyle{agu08}
\bibliographystyle{ieeetr}
\bibliography{carl_abbrev,carl_him,carl_main,carl_calif,carl_alaska,carl_source}
\end{spacing}

\end{enumerate}

%=====================================================
\end{document}
%=====================================================
